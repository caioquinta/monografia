%% ------------------------------------------------------------------------- %%
\chapter{Conclusões}
\label{cap:conclusoes}

\par Um dos fatores principais para o \emph{Lean Startup} ter sido escolhido para direcionar a pesquisa e a elaboração do USP eventos, foi a importância que o cliente tem nesse processo. A fase inicial - de levantamento de interesse por meio de enquete com os alunos - constatou que nem sempre a concepção sobre determinado projeto ou ideia do ponto de visto dos desevolvedores é, de fato, uma necessidade para público-alvo.

\par Nos 3 ciclos de Construir-Medir-Aprender foi possível observar as reações dos usuários e por diversas vezes um comentário ou crítica serviu para modificar uma funcionalidade ou incluir uma nova. Isso demonstra a importância de manter-se o desenvolvimento sempre em contato com o usuário final, para se obter um software que de fato cumpra com sua proposta e seja útil ao usuário. O desenvolvimento em ciclos também mostrou a importância que existe em obter um aprendizado válido para de fato entender a receptividade e necessidade das atualizações.

\par Um desenvolvimento pautado em testes por vezes impediu que um \emph{``bug''} fosse colocado em produção, que apesar de exigir um tempo maior de trabalho, a longo prazo, os testes automatizados geraram ganho em produtividade e eficiência. A integração contínua é uma ferramenta poderosa para situar todos os integrantes do projeto sobre seu estado atual, além de garantir a confiabilidade do sistema.

\par De uma maneira geral foi bastante positivo o resultado final da aplicação. Conseguimos entregar um software sólido, no entanto, apesar de ter sido possível obter \emph{feedbacks} suficientes para melhoria contínua do sistema, infelizmente não tivemos uma adoção de usuários tão grande quanto esperada. Talvez com maior divulgação um público maior pudesse ter aderido e se cadastrado na aplicação.
