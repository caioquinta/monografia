%% ------------------------------------------------------------------------- %%
\chapter{Conclusões}
\label{cap:conclusoes}

\par A escolha do \emph{Lean Startup} deu-se principalmente pela grande importância que é dada para o contato com o cliente e talvez essa tenha sido a principal motivação para escolhermos adotá-la. Foi interessante observar desde a concepção do projeto, ainda durante a fase de enquete, que nem sempre nossa concepção sobre um projeto ou ideia é de fato uma necessidade.

\par Nos 3 ciclos de Construir-Medir-Aprender foi possível observar as reações dos usuários e por diversas vezes um comentário ou crítica serviu para modificar uma funcionalidade ou incluir uma nova. Isso mostra a importância de manter-se o desenvolvimento sempre em contato com o usuário final para se obter um software que de fato cumpra com sua proposta e seja útil para o usuário. O desenvolvimento em ciclos também mostrou a importância que existe em obter um aprendizado válido para de fato entender a receptividade e necessidade das atualizações.

\par Um desenvolvimento pautado em testes por vezes impediu que um \emph{``bug''} fosse colocado em produção, que apesar de exigir um tempo um pouco maior para desenvolver, a longo prazo, os testes automatizados geraram ganho em produtividade e eficiência. A integração contínua é uma ferramenta poderosa para situar todos os integrantes do projeto sobre seu estado atual, além de garantir a confiabilidade do sistema.

\par De uma maneira geral foi bastante positivo o resultado final da aplicação. Conseguimos entregar um software sólido, no entanto, apesar de ter sido possível obter \emph{feedbacks} suficientes para melhorar o sistema, infelizmente não tivemos uma adoção de usuários tão grande quanto esperada. Talvez com maior divulgação um público maior pudesse ter aderido e se cadastrado na aplicação.
