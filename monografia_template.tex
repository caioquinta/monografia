% Arquivo LaTeX de exemplo de monografia para a disciplina MAC0499
%
% Adaptado em julho/2015 a partir do
%
% ---------------------------------------------------------------------------- %
% Arquivo LaTeX de exemplo de dissertação/tese a ser apresentados à CPG do IME-USP
%
% Versão 5: Sex Mar  9 18:05:40 BRT 2012
%
% Criação: Jesús P. Mena-Chalco
% Revisão: Fabio Kon e Paulo Feofiloff


\documentclass[12pt,twoside,a4paper]{book}

% ---------------------------------------------------------------------------- %
% Pacotes
\usepackage[T1]{fontenc}
\usepackage[brazilian]{babel}
\usepackage[utf8]{inputenc}
\usepackage[pdftex]{graphicx}           % usamos arquivos pdf/png como figuras
\usepackage{setspace}                   % espaçamento flexível
\usepackage{indentfirst}                % indentação do primeiro parágrafo
\usepackage{makeidx}                    % índice remissivo
\usepackage[nottoc]{tocbibind}          % acrescentamos a bibliografia/indice/conteudo no Table of Contents
\usepackage{courier}                    % usa o Adobe Courier no lugar de Computer Modern Typewriter
\usepackage{type1cm}                    % fontes realmente escaláveis
\usepackage{listings}                   % para formatar código-fonte (ex. em Java)
\usepackage{titletoc}
%\usepackage[bf,small,compact]{titlesec} % cabeçalhos dos títulos: menores e compactos
\usepackage[fixlanguage]{babelbib}
\usepackage[font=small,format=plain,labelfont=bf,up,textfont=it,up]{caption}
\usepackage[usenames,svgnames,dvipsnames]{xcolor}
\usepackage[a4paper,top=2.54cm,bottom=2.0cm,left=2.0cm,right=2.54cm]{geometry} % margens
%\usepackage[pdftex,plainpages=false,pdfpagelabels,pagebackref,colorlinks=true,citecolor=black,linkcolor=black,urlcolor=black,filecolor=black,bookmarksopen=true]{hyperref} % links em preto
\usepackage[pdftex,plainpages=false,pdfpagelabels,pagebackref,colorlinks=true,citecolor=DarkGreen,linkcolor=NavyBlue,urlcolor=DarkRed,filecolor=green,bookmarksopen=true]{hyperref} % links coloridos
\usepackage[all]{hypcap}                    % soluciona o problema com o hyperref e capitulos
\usepackage[round,sort,nonamebreak]{natbib} % citação bibliográfica textual(plainnat-ime.bst)
\usepackage{emptypage}  % para não colocar número de página em página vazia
\fontsize{60}{62}\usefont{OT1}{cmr}{m}{n}{\selectfont}

% ---------------------------------------------------------------------------- %
% Cabeçalhos similares ao TAOCP de Donald E. Knuth
\usepackage{fancyhdr}
\pagestyle{fancy}
\fancyhf{}
\renewcommand{\chaptermark}[1]{\markboth{\MakeUppercase{#1}}{}}
\renewcommand{\sectionmark}[1]{\markright{\MakeUppercase{#1}}{}}
\renewcommand{\headrulewidth}{0pt}

% ---------------------------------------------------------------------------- %
\graphicspath{{./figuras/}}             % caminho das figuras (recomendável)
\frenchspacing                          % arruma o espaço: id est (i.e.) e exempli gratia (e.g.)
\urlstyle{same}                         % URL com o mesmo estilo do texto e não mono-spaced
\makeindex                              % para o índice remissivo
\raggedbottom                           % para não permitir espaços extra no texto
\fontsize{60}{62}\usefont{OT1}{cmr}{m}{n}{\selectfont}
\cleardoublepage
\normalsize

% ---------------------------------------------------------------------------- %
% Opções de listing usados para o código fonte
% Ref: http://en.wikibooks.org/wiki/LaTeX/Packages/Listings
\lstset{ %
language=Ruby,                  % choose the language of the code
basicstyle=\footnotesize,       % the size of the fonts that are used for the code
numbers=left,                   % where to put the line-numbers
numberstyle=\footnotesize,      % the size of the fonts that are used for the line-numbers
stepnumber=1,                   % the step between two line-numbers. If it's 1 each line will be numbered
numbersep=5pt,                  % how far the line-numbers are from the code
showspaces=false,               % show spaces adding particular underscores
showstringspaces=false,         % underline spaces within strings
showtabs=false,                 % show tabs within strings adding particular underscores
frame=single,                    % adds a frame around the code
framerule=0.6pt,
tabsize=2,                        % sets default tabsize to 2 spaces
captionpos=b,                   % sets the caption-position to bottom
breaklines=true,                % sets automatic line breaking
breakatwhitespace=false,        % sets if automatic breaks should only happen at whitespace
escapeinside={\%*}{*)},         % if you want to add a comment within your code
backgroundcolor=\color[rgb]{1.0,1.0,1.0}, % choose the background color.
rulecolor=\color[rgb]{0.8,0.8,0.8},
extendedchars=true,
xleftmargin=10pt,
xrightmargin=10pt,
framexleftmargin=10pt,
framexrightmargin=10pt
}

% ---------------------------------------------------------------------------- %
% Corpo do texto
\begin{document}

\frontmatter
% cabeçalho para as páginas das seções anteriores ao capítulo 1 (frontmatter)
\fancyhead[RO]{{\footnotesize\rightmark}\hspace{2em}\thepage}
\setcounter{tocdepth}{2}
\fancyhead[LE]{\thepage\hspace{2em}\footnotesize{\leftmark}}
\fancyhead[RE,LO]{}
\fancyhead[RO]{{\footnotesize\rightmark}\hspace{2em}\thepage}

\onehalfspacing  % espaçamento

% ---------------------------------------------------------------------------- %
% CAPA
\thispagestyle{empty}
\begin{center}
    \vspace*{2.3cm}
    Universidade de São Paulo\\
    Instituto de Matemática e Estatística\\
    Bacharelado em Ciência da Computação


    \vspace*{3cm}
    \Large Caio Teixeira da Quinta \\ Eugênio Augusto Jimenes


    \vspace{3cm}
    \textbf{\Large{Plataforma Web para divulgação e centralização de eventos aplicando conceitos de Métodos Ágeis \\ e Lean Startup }}


    \vskip 5cm
    \normalsize{São Paulo}

    \normalsize{Dezembro de 2016}
\end{center}

% ---------------------------------------------------------------------------- %
% Página de rosto
%
\newpage
\thispagestyle{empty}
    \begin{center}
        \vspace*{2.3 cm}
        \textbf{\Large{Plataforma Web para divulgação e centralização de eventos aplicando conceitos de Métodos Ágeis \\ e Lean Startup }}
        \vspace*{2 cm}
    \end{center}

    \vskip 2cm

    \begin{flushright}
    Monografia final da disciplina \\
        MAC0499 -- Trabalho de Formatura Supervisionado.
    \end{flushright}

    \vskip 5cm

    \begin{center}
    Supervisor: Prof. Dr. Alfredo Goldman vel Lejbman\\
    Cosupervisor: Jorge Melegati

    \vskip 5cm
    \normalsize{São Paulo}

    \normalsize{Dezembro de 2016}
    \end{center}
\pagebreak



\pagenumbering{roman}     % começamos a numerar

%% % ---------------------------------------------------------------------------- %
%%Agradecimentos:
%% % Se o candidato não quer fazer agradecimentos, deve simplesmente eliminar esta página
\chapter*{Agradecimentos}
%% Texto texto texto texto texto texto texto texto texto texto texto texto texto
%% texto texto texto texto texto texto texto texto texto texto texto texto texto
%% texto texto texto texto texto texto texto texto texto texto texto texto texto
%% texto texto texto texto. Texto opcional.



% ---------------------------------------------------------------------------- %
% Resumo
\chapter*{Resumo}
    \par Este referencial teórico tem por objetivo fundamentar a plataforma USP Eventos. A ferramenta foi criada a partir da constatação de que a Cidade Universitária possui grande diversidade acadêmica e cultural que manifesta-se em uma variedade de eventos que são promovidos e realizados em toda sua extensão. Levando em consideração essa grande quantidade de eventos é esperado que existam problemas para sua divulgação. Ao realizar uma enquete junto à comunidade USP constatou-se a necessidade de um sistema para centralizar a divulgação desses eventos. Como consequência desse resultado foi proposto criar o USP Eventos. Para auxiliar no desenvolvimento do sistema foi utilizada uma abordagem que combinou a metodologia \emph{Lean Startup} com conceitos de Métodos Ágeis. Ao longo do processo foi possível observar os benefícios e desvantagens das metodologias escolhidas e desenvolver um sistema direcionado aos interesses dos usuários.
\\

\noindent \textbf{Palavras-chave:} eventos, métodos ágeis, lean startup, desenvolvimento web.

% ---------------------------------------------------------------------------- %
% Abstract
\chapter*{Abstract}

Elemento obrigatório, elaborado com as mesmas características do resumo em
língua portuguesa.
\\

\noindent \textbf{Keywords:} keyword1, keyword2, keyword3.


% ---------------------------------------------------------------------------- %
% Sumário
\tableofcontents    % imprime o sumário




%% % ---------------------------------------------------------------------------- %
\chapter{Lista de Abreviaturas}
\begin{tabular}{ll}
            MVP         &  Produto Mínimo Viável (\emph{ Minimum Viable Product})\\
            CoC         &  Convenção sobre Configuração (\emph{ Convention over Configuration}) \\
            DRY         &  Não se repita (\emph{Don't Repeat yourself})\\
            ORM         &  Mapeamento Objeto Relacional(\emph{Object Relational Mapping})\\
            POP         &  Painel de Opinião Popular\\
            UX          &  Experiência do Usuário ( \emph{User Experience})\\
            UI          &  Interface de Usuário (\emph{User Interface})\\
            RoR         &  Ruby on Rails\\
            Pass        &  Plataforma como serviço (\emph{Platform-as-a-Service})\\
            GA          & Google Analytics \\
            SO          & Sistema Operacional \\
%%          CFT         & Transformada contínua de Fourier (\emph{Continuous Fourier Transform})\\
%%          DFT         & Transformada discreta de Fourier (\emph{Discrete Fourier Transform})\\
%%         EIIP         & Potencial de interação elétron-íon (\emph{Electron-Ion Interaction Potentials})\\
%%         STFT         & Transformada de Fourier de tempo reduzido (\emph{Short-Time Fourier Transform})\\
\end{tabular}

%% % ---------------------------------------------------------------------------- %
%% \chapter{Lista de Símbolos}
%% \begin{tabular}{ll}
%%         $\omega$    & Frequência angular\\
%%         $\psi$      & Função de análise \emph{wavelet}\\
%%         $\Psi$      & Transformada de Fourier de $\psi$\\
%% \end{tabular}

%% % ---------------------------------------------------------------------------- %
%% % Listas de figuras e tabelas criadas automaticamente
%% \listoffigures
%% \listoftables



% ---------------------------------------------------------------------------- %
% Capítulos do trabalho
\mainmatter

% cabeçalho para as páginas de todos os capítulos
\fancyhead[RE,LO]{\thesection}

\singlespacing              % espaçamento simples
%\onehalfspacing            % espaçamento um e meio


\input cap-introducao       % associado ao arquivo: 'cap-introducao.tex'
\input cap-leanstartup      % associado ao arquivo: 'cap-leanstartup.tex'
\input cap-metodosageis   % associado ao arquivo: cap-leanstartup.tex
\input cap-tecnologias         % associado ao arquivo: 'cap-tecnologias.tex'
\input cap-uspeventos        % associado ao arquivo: 'cap-uspeventos.tex'
\input cap-conclusoes       % associado ao arquivo: 'cap-conclusoes.tex'

% cabeçalho para os apêndices
\renewcommand{\chaptermark}[1]{\markboth{\MakeUppercase{\appendixname\ \thechapter}} {\MakeUppercase{#1}} }
\fancyhead[RE,LO]{}
\appendix

\chapter{Título do apêndice}
\label{cap:ape}

Texto texto texto texto texto texto texto texto texto texto texto texto texto
texto texto texto texto texto texto texto texto texto texto texto texto texto
texto texto texto texto texto texto.
      % associado ao arquivo: 'ape-conjuntos.tex'


% ---------------------------------------------------------------------------- %
% Bibliografia
\backmatter \singlespacing   % espaçamento simples
\bibliographystyle{plainnat-ime} % citação bibliográfica textual
\bibliography{bibliografia}  % associado ao arquivo: 'bibliografia.bib'


%%%  ---------------------------------------------------------------------------- %
%% % Índice remissivo
%% \index{TBP|see{periodicidade região codificante}}
%% \index{DSP|see{processamento digital de sinais}}
%% \index{STFT|see{transformada de Fourier de tempo reduzido}}
%% \index{DFT|see{transformada discreta de Fourier}}
%% \index{Fourier!transformada|see{transformada de Fourier}}

%% \printindex   % imprime o índice remissivo no documento

\end{document}
