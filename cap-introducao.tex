%% ------------------------------------------------------------------------- %%
\chapter{Introdu��o}
\label{cap:introducao}
\section{Motiva��o e Objetivos}
\par � not�vel que a Cidade Universit�ria possui uma imensa diversidade acad�mica e cultural que se manifesta em  uma variedade de eventos sendo promovidos e organizados pela Universidade de S�o Paulo al�m de iniciativas da pr�pria comunidade para ocupar o espa�o p�blico.

\par Ao realizar uma enquete junto � comunidade USP foi perebido um v�cuo na divulga��o de eventos dentro do campus pois devido a sua extens�o e diversidade a propaga��o de informa��es muitas vezes dispersava-se levando os usu�rios a perder oportunidades interessantes por n�o tomar conhecimento � tempo.

\par Em vista desse problema foi proposto criar um Sistema Web colaborativo para centralizar e divulgar os eventos que s�o organizados pela Universidade de S�o Paulo assim como pela comunidade que frequenta o campus e para isso adotamos uma metodologia de projeto seguindo os prececeitos de Lean Startup e M�todos �geis.

\par A escolha da metodologia Lean Startup deu-se pela sua ampla aplica��o em projetos com grande grau de incerteza no qual uma abordagem interativa e um desenvolvimento incremental auxiliam no aprendizado e direcionam o desenvolvimento. Durante o projeto descrevemos como tais conceitos auxiliaram no desenvolvimento do sistema e as vantagens dessa abordagem.

\section{Cap�tulos}
