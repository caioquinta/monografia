%% ------------------------------------------------------------------------- %%
\chapter{Introdução}
\label{cap:introducao}
\section{Motivação e Objetivos}

\par A Cidade Universitária possui uma variedade de eventos sociais e acadêmicos, que ocorrem, por vezes, simultâneamente em toda sua extensão. Esse cenário se dá pela complexidade cultural existente no campus, que envolve discentes, docentes e comunidade.

\par Junto à comunidade USP, foi realizada uma enquete via e-mail, na qual constatou-se o interesse em uma plataforma para divulgar e centralizar os inúmeros eventos, acadêmicos ou não, da Cidade Universitária. Para atender a essa demanda de interesse, foi proposta a criação do USP Eventos, um sistema web voltado para os usuários se informarem sobre o que ocorre no campus, além de incentivar a organização de eventos de modo a ocupar o espaço público.

\par Para o desenvolvimento do projeto foi escolhida uma abordagem baseada em conceitos de \emph{Lean Startup} e Métodos Ágeis. A escolha dessa metodologia combinada dá-se pela sua ampla aplicação em projetos com grande grau de incerteza, nos quais um tratamento interativo e um desenvolvimento incremental auxiliam no aprendizado sobre os interesses do usuários.

\par O objetivo do trabalho foi aplicar os conceitos de \emph{Lean Startup} e Métodos Ágeis em um projeto prático a fim de desenvolver um software consistente que atendesse às necessidades dos usuários, além de possibilitar observar as vantagens e desvantagens das abordagens escolhidas.
