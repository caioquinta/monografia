%% ------------------------------------------------------------------------- %%
\chapter{Introdução}
\label{cap:introducao}
\section{Motivação e Objetivos}

\par Devido a sua diversidade cultural a Cidade Universitária possui uma grande abundância de eventos sociais acadêmicos ocorrendo em seu dia-a-dia sendo organizados pela própria Universidade ou pelos próprios usuários do campus.

\par Ao realizar uma enquete junto à comunidade USP foi descoberto o interesse em uma plataforma para centralizar os eventos na USP pois devido a extensão física do campus e da diversidade de eventos, a propagação de informações muitas vezes é dispersa levando os usuários a perder oportunidades interessantes por não tomar conhecimento à tempo.

\par Em vista desse problema foi proposto criar um Sistema Web colaborativo para centralizar e divulgar os eventos que são organizados pela Universidade de São Paulo e comunidade para isso adotamos uma metodologia de projeto seguindo os preceitos de Lean Startup e Métodos Ágeis.

\par A escolha da metodologia Lean Startup deu-se pela sua ampla aplicação em projetos com grande grau de incerteza no qual uma abordagem interativa e um desenvolvimento incremental auxiliam no aprendizado e direcionam o desenvolvimento. Durante o projeto descrevemos como tais conceitos auxiliaram no desenvolvimento do sistema e as vantagens dessa abordagem.

\section{Capítulos}
