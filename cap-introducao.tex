%% ------------------------------------------------------------------------- %%
\chapter{Introdução}
\label{cap:introducao}
\section{Motivação e Objetivos}

\par Devido à sua diversidade cultural, a Cidade Universitária possui grande abundância de eventos sociais e acadêmicos ocorrendo em toda sua extensão. 

\par Ao realizar uma enquete junto à comunidade USP foi constatado interesse em uma plataforma para divulgar e  centralizar tais eventos. 

\par Como resultado foi proposto criar o USP Eventos, um sistema web voltado para os usuários se informarem sobre o que ocorre no campus, além de incentivar a organização de eventos de modo a ocupar o espaço público.

\par Para o desenvolvimento do projeto foi escolhida uma abordagem baseada em conceitos de Lean Startup e Métodos Ágeis.

\par  A escolha da metodologia Lean Startup combinada com conceitos de Métodos Ágeis deu-se pela sua ampla aplicação em projetos com grande grau de incerteza, nos quais uma abordagem interativa e um desenvolvimento incremental auxiliam no aprendizado sobre os interesses do usuários.

\par O objetivo do trabalho foi aplicar os conceitos de Lean Startup e Métodos Ágeis em um projeto prático a fim de desenvolver um software consistente que atendesse às necessidades dos usuários, além de possibilitar observar as vantagens e desvantagens das abordagens escolhidas.
