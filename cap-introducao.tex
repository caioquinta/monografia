%% ------------------------------------------------------------------------- %%
\chapter{Introdu��o}
\label{cap:introducao}
\section{Motiva��o e Objetivos}
\par � not�vel que a Cidade Universit�ria possui uma imensa diversidade acad�mica e cultural que se manifesta em  uma variedade de eventos sendo promovidos e organizados pela Universidade de S�o Paulo al�m de iniciativas da pr�pria comunidade para ocupar o espa�o p�blico.

\par Levando em considera��o a quantidade de eventos acontecendo de forma simult�nea, a extens�o f�sica do campus, a pulveriza��o dos eventos espalhando-se por toda sua extens�o  e a quantidade de informa��es dispersas entre as v�rias redes de comunica��o oficiais ou n�o  muitas vezes n�o tomamos conhecimento a tempo de alguma oportunidade que poderia ser interessante.

\par Foi proposto criar um Sistema Web colaborativo para centralizar e divulgar os eventos que s�o organizados pela Universidade de S�o Paulo assim como pela comunidade que frequenta o campus e para isso adotamos uma metodologia de projeto seguindo os prececeitos de Lean Startup e M�todos �geis.

\par A metodologia Lean Startup tem como foco evitar o desperd�cio de tempo e recursos focando em um r�pido aprendizado dos interesses do p�blico alvo para refinamento do produto portanto foi adotada no desenvolvimento do USP Eventos pois dada sua natureza colaborativa do projeto era vital podermos coletar informa��es sobre o nosso p�blico, a comunidade usu�ria do campus da Cidade Universit�ria, a fim de que o projeto estivesse alinhado com seus interesses.

\par Para atingir os objetivos propostos pela metodologia Lean Startup  aplicamos conceitos de M�todos �geis tais como testes, desenvolvimento interativo, incremental e cont�nuo. 

\section{Cap�tulos}
