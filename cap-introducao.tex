%% ------------------------------------------------------------------------- %%
\chapter{Introdu��o}
\label{cap:introducao}
\section{Motiva��o e Objetivos}

\par Devido a sua grande diversidade cultural a Cidade Universit�ria possui uma grande abund�ncia de eventos sociais acad�micos ocorrendo em seu dia-a-dia sendo organizados pela pr�pria Universidade ou pelos pr�prios usu�rios do campus.

\par Ao realizar uma enquete junto � comunidade USP foi percebido o interesse em uma plataforma para centralizar os eventos na USP pois devido a extens�o f�sica do campus e diversidade de eventos propaga��o de informa��es muitas vezes dispersava-se levando os usu�rios a perder oportunidades interessantes por n�o tomar conhecimento � tempo.

\par Em vista desse problema foi proposto criar um Sistema Web colaborativo para centralizar e divulgar os eventos que s�o organizados pela Universidade de S�o Paulo e comunidade para isso adotamos uma metodologia de projeto seguindo os prececeitos de Lean Startup e M�todos �geis.

\par A escolha da metodologia Lean Startup deu-se pela sua ampla aplica��o em projetos com grande grau de incerteza no qual uma abordagem interativa e um desenvolvimento incremental auxiliam no aprendizado e direcionam o desenvolvimento. Durante o projeto descrevemos como tais conceitos auxiliaram no desenvolvimento do sistema e as vantagens dessa abordagem.

\section{Cap�tulos}
