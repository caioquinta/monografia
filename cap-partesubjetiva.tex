\chapter{Parte Subjetiva}
\label{cap:partesubjetiva}

\section{Desafios}
\par Sendo um trabalho que requer dedicação constante, devido aos ciclos de interação, muitas vezes se fez necessário conciliar o tempo de dedicação ao projeto com trabalho e provas de outras disciplinas. A organização pessoal do tempo foi bastante importante e difícil de conciliar.
\par Havia um conhecimento prévio por parte dos integrantes do arcabouço Ruby on Rails, o que possibilitou menor tempo dedicado ao aprendizado da tecnologia e facilitou o desenvolvimento, porém, muitas vezes levou-se mais tempo construindo os testes automatizados do que a funcionalidade em si. Não foram poucas as vezes em que foi difícil manter-se focado na abordagem utilizando testes, o que exigiu grande autodisciplina. No entanto, essa postura mostrou-se útil e de fato recompensadora.
\par Devido à ausência de disciplinas focadas em interface de usuário, em dado ponto do projeto foi necessário realizar uma pesquisa sobre UX/UI. Sendo uma área bastante fora do comum dentro da formação tradicional do curso, houve aprendizado enorme nessa fase. Houve ainda grande dificuldade para divulgar a plataforma. pois os canais de comunicação, apesar de variados, não parecem ter sido suficientes.
\par Um ponto crucial para o sucesso do projeto foi a utilização do Kanban, pois em um trabalho extenso era fácil ver-se sobrepujado pela quantidade de tarefas a serem desempenhadas. Manter um quadro Kanban com tarefas curtas e pontuais ajudou a manter o foco e a organização.

\section{Disciplinas Relevantes}
\begin{itemize}
\item\textbf{MAC110, MAC121, MAC323, MAC211, MAC242:}
\par O conteúdo dessas disciplinas foi o primeiro contato com programação propriamente dita, sendo fundamental para a formação como programador.
\item\textbf{MAC0426 – Banco de Dados e MAC0439 – Laboratório de Banco de Dados:}
\par Talvez pudessem ser uma única disciplina, porém o conhecimento em modelagem de banco de dados e sistemas de gerenciamento são fundamentais para a formação como programador.
\item\textbf{MAC0441 – Programação Orientada à Objetos:}
\par Sem dúvida uma disciplina optativa que deveria ser obrigatória, por ser um dos paradigmas mais utilizados atualmente.
\item\textbf{MAC0316 – Conceitos Fundamentais de Linguagens de Programação:}
\par Aprender um paradigma completamente diferente como programação funcional foi um desafio enorme, porém muito importante para formação.
\item\textbf{MAC0446 – Princípios de Interação Humano-Computador:}
\par Talvez a única disciplina dentro do programa com enfoque no usuário e na sua interface.
\item\textbf{MAC0458: Direito e Software Livre:}
\par Uma disciplina bastante interessante por explicar e fomentar uma discussão a respeito de Software livre e as questões éticas e legais envolvidas. Deveria ser obrigatória dentro do currículo.
\end{itemize}

\section{Próximos Passos}
\par O sistema ainda tem muito espaço para evolução. Durante a última interação foram recebidos uma série de \emph{feedbacks} e opções para continuar o desenvolvimento do software:
\begin{itemize}
\item{Integrar o sistema com o Google Agenda para notificar usuários dos seus eventos de interesse;}
\item{Criar uma página contendo um Feed de notícias com as novidades ocorrendo no Campus;}
\item{Opção de Login com Instagram, Twitter e outras redes sociais;}
\item{Remodelar a página de Alertas com um \emph{layout} mais atrativo}
\end{itemize}
