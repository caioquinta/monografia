%% ------------------------------------------------------------------------- %%
\chapter{Lean Startup}
\label{cap:leanstartup}
\section{Lean Startup: O que é}
\subsection{Startup: uma definição}
\par Através da popularização da internet e dos computadores pessoais nos anos 90, nos Estados Unidos, o termo startup foi generalizado para classificar pequenas empresas com propostas inovadoras, sejam por atuarem com as novas tecnologias que surgiram para o grande mercado na época como as chamadas empresas online ou empresas "ponto com" ou  pelo seu novo modo de organização e processo de produção.
\par \cite{nicolo:14} provê uma definição de uma startup de software baseada nos desafios que ela enfrenta:
\begin{itemize}
\item Pouco ou nenhum histórico operacional - Startups com pouca ou nenhuma experiência em desenvolver processos de negócio e gerenciamento organizacional.
\item Recursos limitados - Startups tipicamente focam em lançar um produto, promovê-lo e construir alianças estratégicas.
\item Múltiplas Influências - Pressão dos investidores, clientes, parceiros e competidores impactam nas tomadas de decisões de uma empresa. Apesar de importantes em média elas tendem a ser inconsistentes.
\item Mercado e tecnologias dinâmicas- Empresas de softwares novos frequentemente precisam desenvolver ou operar com tecnologias disruptivas para atuar em potenciais mercados alvos.
\end{itemize}

\par Apesar de não ser um rótulo exclusivo para o mercado de Tecnologia da Informação alguns locais e atividades foram particularmente associados à classificação devido a revolução tecnológica promovida pela bolha da internet, como no Vale do Silício, área norte do estado da Califórnia, EUA, conhecida até hoje por ser um ecossistema constituído de empresas inovadoras.
\par Com o passar dos anos e com o impacto da internet no mercado global o termo amadureceu para empresa, grupo ou organização que busca um modelo de negócios escalável geralmente envolvida em implementações de processos inovadores de desenvolvimento e pesquisa de mercado-alvo. Grosseiramente uma startup surge sob uma ideia ou foco que pode ou não dar certo mas consciente desta instabilidade.
\par Para Steve Blank, acadêmico e empreendedor do Vale do Silício, uma startup vai de fracasso à fracasso com objetivo de aprender com cada falha e assim definir o que não funciona no processo na qual a empresa esta engajada. Por isso é considerado um modelo de negócio escalável: deve ser flexível perante a constante reação do mercado e da própria produção.

\subsection{Lean Startup}
\par \emph{Lean Startup} (Startup Enxuta) é um conceito introduzido por Eric Ries ( \cite{ries:11}), empreendedor de diversas startups do Vale do Silício.
\par Trata-se de uma metodologia de negócios derivada da combinação de outros padrões de desenvolvimento como \emph{Minimal Viable Product} (produto viável mínimo), \emph{Customer Development} (desenvolvimento voltado ao cliente) e \emph{Agile software development} (desenvolvimento ágil de software ou Método Ágil).
\par Ries propõe que é possível encurtar os ciclos de implementação de um produto (ou solução) adotando uma combinação de testes, hipóteses de negócio e degustações por parte do público-alvo.
\par Através do lançamento periódico de cada versão do produto é possível avaliar não apenas quesitos técnicos como também a reação do mercado. Consequentemente o retorno de cada iteração afeta o planejamento do produto e suas futuras versões. 
\par Essa economia em cada ciclo provém do \emph{validated learning} (aprendizagem de validação), ou seja, toda ideia, funcionalidade ou vertente de produto deve ser primeiramente experimentada e testada.
\par Isso evita desperdício de desenvolvimento além de  abrir oportunidades para adaptações e alterações de projeto em casos de falha ou rejeição do cliente.
\par Versões simples do produto sob cada ciclo de avaliação é uma estratégia derivada do padrão de Mínimo Produto Viável( \emph{Minimal Viable Product}), proposta em 1996 por Frank Robinson, CEO da empresa SyncDev, porém popularizado anos depois por Steve Blank.
\par Robinson propõe o lançamento de uma versão o mais simples possível de modo à antecipar a análise de mercado e assim minimizar o risco de retorno por parte da empresa. A popularidade Steve Blank foi em adaptar a estratégia incluindo o lado do cliente na balança, o que ele chama de Customer Development. Blank vai além de apenas minimizar o risco de retorno: busca compreender as necessidades do cliente. 
\par O \emph{Lean Startup} aprimora ainda mais o conceito para avaliações sob cada interação e assim maximizar o aprendizado e evolução do projeto alinhado com o desejo do mercado. Esse cliclo de evolução e aprendizado é  chamado Construir-Medir-Aprender (\emph{Build-Measure-Learn}).

\section{Produto Mínimo Viável (MVP)}

\par Uma definição de Produto Mínimo Viável por Eric Ries (\cite{ries:11}):
\emph{"A Minimum Viable Product is that version of a new product which allows a team to collect the maximum amount of validated learning about customers with the least effort."}.
    \par A idéia central do conceito de MVP é maximizar a validação de aprendizado sobre um produto utilizando o menor esforço possível. Como na maioria das startups os recursos financeiros e humanos são bastante escassos o tempo de validação do produto e determinação do interesse do público é um fator decisivo para o sucesso da mesma.
    \par Um MVP deve possuir 3 características \footnote{ Retirado de Techopedia: Minimum Viable Product (MVP): \url{https://www.techopedia.com/definition/27809/minimum-viable-product-mvp}}.:
    \begin{enumerate}
        \item Ter valor suficiente para que uma pessoa queira utilizá-lo ou comprá-lo.
        \item Possui suficientes benefícios para reter os chamados usuários pioneiros (emph{early adopters}) \footnote{Os primeiros consumidores de um produto que acaba de tornar-se disponível}.
        \item Ser capaz de prover um ciclo de \emph{feedback} suficiente para guiar o desenvolvimento.
\end{enumerate}
    \par Durante a concepção do projeto são definidas algumas hipóteses sobre o produto e na etapa do MVP é definido então qual será o seu núcleo, ou seja, quais funcionalidades ou estratégias queremos testar de modo que possamos validar  as nossas hipóteses iniciais e obter o máximo de aprendizado possível utilizando-o para planejar novas funcionalidades e determinar as prioridades para a equipe de desenvolvimento.
    \par Além da validação de aprendizado outra vantagem significativa provida pelo MVP é evitar desperdícios. 
    \par O MVP permite testar se a funcionalidade ou hipótese sobre um projeto é bem aceita pelo público alvo implementando-a de uma forma simplificada sem dispender horas a fio de desenvolvimento.
    \par Dessa forma caso comprove-se que tal premissa não é interessante para o projeto seu desenvolvimento é interrompido sem que tenham sido desperdiçados tempo e recursos.
    \par Os termos "mínimo" e "máximo" refêrentes a "máximo apredizado" e "mínimo produto víavel" frequentemente se mostram vagos na documentação do que é um MVP, citando Eric Ries: \emph{"[...]the definition's use of the words maximum and minimum means it is decidedly not formulaic. It requires judgment to figure out, for any given context, what MVP makes sense. "} \footnote{Fonte: Startup Lessons Learned, Minimum Viable Product: a guide \url{http://www.startuplessonslearned.com/2009/08/minimum-viable-product-guide.html}}. É importante ressaltar que um MVP não é um produto completo com as funcionalidades  mínimas e sim um conjunto de características mínimas que configuram o serviço ou produto que está sendo oferecido. Dessa forma um MVP pode ser apenas um protótipo, um produto completo ou mesmo apenas um \emph{mock-up} do que será oferecido na versão completa.
    \par Alguns tipos de MVP são: \footnote{ Fonte: Scale my Business: The Ultimate Guide to Minimum Viable Products  \url{http://scalemybusiness.com/the-ultimate-guide-to-minimum-viable-products/}}
\begin{itemize}
\item \emph{ Vídeo Explicativo:}
Um vídeo curto contendo uma explicação clara do que o seu produto faz e porque as pessoas deveriam utilizá-lo. Esse é o caso do dropbox que fez um video \footnote{Link para o vídeo  \url{https://www.youtube.com/watch?v=7QmCUDHpNzE}} com cerca de 5 minutos explicando o que era o seu serviço.
\item \emph{Landing Page:}
Criar uma página inicial contendo uma explicação detalhada do que é o produto que você irá oferecer juntamente com um formulário de contato. Através de uma configuração simples pelo Google Analytics é possível manter um registro de conversões (no caso cadastros do formulário)  a fim de medir o interesse das pessoas no seu produto.
\item \emph{MVP "Mago de OZ":}
A idéia é criar uma página visualmente completa que funcione como o produto final mas que na verdade  exista alguém  executando as tarefas manualmente. Esse foi o caso da Zappos hoje a maior vendedora de sapatos dos Estados Unidos.
\item \emph{ MVP com Consierge:}
Ao invés de prover um produto você realiza manualmente o serviço executando
exatamente os mesmos passos para o usuário que o a sua empresa realizaria. É um método não escalável e lento para executar pois requer que você esteja em contato direto com o cliente e realize as tarefas manualmente porém isso permite um rápido aprendizado sobre o produto e o cliente.
\\A empresa Food on the Table ajuda seus consumidores a criarem listas de compras, acharem receitas e conseguirem descontos nos ingredientes em seus supermercados favoritos, inicialmente seus fundadores encontraram uma senhora interessada no serviço e por 10 dólares/semana eles mantinham as listas de compra e procurava, por descontos nos supermercados em que ela fazia  compras.
\end{itemize}

\section{O ciclo de Build-Measure-Learn}

    \par Um ciclo de build measure learn é uma abordagem de desenvolvimento do produto que aprimora  o tradicional modelo de desenvolvimento "Cascata" utilizado de forma abrangente durante o século XX.
\subsection{O modelo Cascata}
    \par O nome "Cascata" é bastante literal e tem origem na própria estrutura do método (figura \ref{fig:waterfall}) que seguia uma série de passos de forma sequencial de maneira bastante direta, como se fosse uma cascata.
\begin{figure}[htb]
\centering
\includegraphics[width=10cm]{figuras/waterfall}
\caption{\label{fig:waterfall}O Modelo Cascata.}
\end{figure}
\par O método consiste em um desenvolvimento sequencial e não-iterativo inicializando com a parte de análise de requisitos e de forma subsequente temos as etapas de design de projeto, implementação, verificação e manutenção.

Uma rápida explicação dos passos:
\begin{itemize}
\item \emph{ Requerimentos:} Realizar a análise de requisitos do projeto.
\item \emph{ Design de Projeto:}  Focando na estrutura de dados, arquitetura do software, detalhes procedais e caracterização das interfaces é formulado um documento de forma a apresentar os requerimentos de uma forma que possa ser interpretado pelos programadores.
\item \emph{ Implementação:} Etapa da codificação do projeto propriamente dita.
\item \emph{  Verificação: } Etapa para teste do produtos visando eliminar qualquer \emph{bug} que possa ter passado despercebido e refinar a lógica interna do sofware caso necessário.
\item \emph{ Manutenção: } Etapa para instalação do sistema no cliente, configuração de servidores, etc.
\end{itemize}

\par Uma das grandes das críticas dessa abordagem é que dificilmente um desenvolvimento de software segue todas as etapas da forma como o modelo propõe e nem sempre o cliente sabe definir bem os requisitos antes de ver o software funcionando resultando em tempo e desenvolvimento desperdiçado em funcionalidades que não resolvem o problema além de mudanças tardias no escopo do projeto  que encarecem o custo total e poderiam ter sido evitadas e contornadas de maneira mais satisfatória  em um modelo com um processo de desenvolvimento iterativo \footnote{ Fonte: Wikipedia - Waterfall Model \url {https://en.wikipedia.org/wiki/Waterfall_model}}.
\par Ao contrário do modelo cascasta no qual o contato do cliente com o software se dá apenas no fim do processo de desenvolvimento na etapa de testes o método de \emph{Build-Measure-Learn} privilegia desde o início uma interação contínua com o cliente.

\subsection{Build-Measure-Learn (Construir-Medir-Aprender)}
\par Na fase de Verificação do modelo Cascata existia a possibilidade de disponibilizar para os clientes versões alphas ou betas do sofware em questão, como o foco não era obter um retorno sobre o desenvolvimento e sim verificar a existência de \emph{bugs} o usuário acabava completamente fora do processo de desenvolvimento.
\par Com o surgimento da Metodologia Ágil foi possível criar softwares de maneira interativa e envolver o cliente no processo porém devido a falta de um arcabouço para testar às hipóteses comerciais acabava-se muitas vezes por desenvolver um software com todas as funcionalidades que o cliente gostaria mas não obter um sucesso comercial.\footnote{Fonte: Por Steve Blank em \url{http://venturebeat.com/2015/05/06/build-measure-learn-doesnt-mean-throwing-things-against-the-wall-to-see-if-they-stick/}}
\par O modelo de Construir-Medir-Aprender surge então com o principal objetivo de eliminar as incertezas sobre as hipóteses do produto.
\par  Através de um aprendizado rápido sobre o comportamento dos usuários é possível  minimizar os riscos  e custos desnecessários que persistir em uma idéia equivocada pode causar, mantendo o aspecto interativo presente na metodologia Ágil e obtendo um aprendizado sobre o comportamento do usuário a cada interação.
\par O ciclo de \emph{Build-Measure-Learn} é uma das idéias fundamentais do \emph{Lean Startup} consistindo de um ciclo de 3 fases (Construir, Medir e Aprender) porém para melhor ilustrar o que representa cada uma das fases foi incluído na figura \ref{fig:buildmeasurelearn} outras 3 menores (\emph{ideas, data e code}).
\begin{figure}[htb]
\centering
\includegraphics[width=10cm]{figuras/buildmeasurelearn}
\caption{\label{fig:buildmeasurelearn}O ciclo de Build Measure Learn.}
\end{figure}
\begin{itemize}
\item \emph{Build:} Inicialmente temos algumas idéias que foram definidas a partir das hipóteses do produto que precisam ser implementadas (\emph{code}) no MVP.
\item \emph{Measure:} Implementado o MVP coletamos dados(\emph{data}) de uso e seguindo algumas métricas já definidas avaliamos seu desempenho. Todo o ciclo é baseado na idéia de aprendizado rápido para aprender o máximo possível sobre como os usuários reagem as idéias implementadas.
\item \emph{Learn:} A partir então da análise dos dados coletados podemos inferir sobre continuar sobre como continuar o desenvolvimento e o que funcionou ou não. Feito isso implementamos as novas idéias geradas e começamos novamente o ciclo para validá-las.
\end{itemize}
\par As etapas do ciclo não precisam necessariamente ocorrer em ordem podendo se sobrepor  ou mesmo serem unidas dependendo de como for o ciclo de desenvolvimento. (\cite{ries:11}) 
\par As alterações de software precisam ser feitas de maneira rápida de modo a testar o mais rápido possível novas idéias, portanto é importante  que as funcionalidades devem se manter simples e diretas pois o foco é o aprendizado gerado e não desenvolver um software ou um protótipo completo.
\par Dessa maneira é possível evitar que sejam desperdiçados recursos em uma funcionalidade que pode não ser bem recebida abandonando seu desenvolvimento.
\par Para minimizar que um sistema com problemas seja colocado em produção e acelerar o processo de desenvolvimetno procura-se utilizar ferramentas que auxiliam na integração contínua do software além de executar testes automatizados.
\par A utilização desses recursos permite seja possível manter um desenvolvimento consistente e confiável sem comprometer o tempo de execução.
\par O que o Construir-Medir-Aprender perde de vista é que novos empreendimentos, tanto startups quanto novas ideias dentro de empresas já existentes não começam com ideias mas com hipóteses. 
\par "Ideia" evoca uma visão que imediatamente requer um plano para se frutificar. Em contraste, "hipótese" indica um palpite com precedentes que requer experimentação e dados para ser validado ou invalidado. (\cite{blankendeavor} ).
\par Como o que você constrói deve estar alinhado com o as hipóteses formuladas e a cada ciclo é necessário sempre testar novas hipóteses resultando em diferentes protótipos a figura \ref{fig:hypotheses-experiment} representa uma variação do ciclo de Construir-Medir-Aprender cuja proposta é enfatizar quais hipóteses deveriam ser testadas.
\begin{figure}[htb]
\centering
\includegraphics[width=5cm]{figuras/hypotheses-experiment}
\caption{\label{fig:hypotheses-experiment} Uma variação para o ciclo de Construir-Medir-Aprender.}
\end{figure}
\par Adquirindo os benefícios herdados pela Metodologia Ágil e somando-se a isso um arcabouço para adquirir aprendizado sobre a utilização do software o ciclo de Construir-Medir-Aprender resulta em um sofware mais refinado e alinhado com as expectativas dos clientes.

\section{Desenvolvimento de Clientes}
TODO
