%% ------------------------------------------------------------------------- %%
\chapter{Usp Eventos}
\label{cap:uspeventos}
\section{Definição do Projeto}
\subsection{Motivação}
        \par A idéia de desenvolver um sistema utilizando Métodos Agéis e conceitos de Lean Startup surgiu em dezembro de  2015.
	\par O objetivo era  desenvolver um sistema web ou aplicativo voltado para a comunidade USP com a intenção de facilitar de alguma forma o dia-a-dia dos usuários.
        \par Inicialmente existiam algumas propostas de projeto que foram então formalizadas em uma enquete realizada junto a comunidade USP.
\subsection{Enquete e definição do projeto}
        \par  No início as seguintes hipóteses de interesse de projeto foram disponibilizadas para votação:
                \begin{itemize}
                \item {Usp avisa eventos e incidentes:} Um sistema para reportar desde eventos acontecendo no campus (palestras, festas, etc) até outros incidentes (buracos, perigos, etc).
                \item {Usp doações e trocas: } Um sistema voltado para os membros da comunidade que desejam fazer doações ( sejam móveis, roupas usadas, etc) para outros e aqueles dispostos a receber. organizando filas de interesse,disponibilidade e urgência.
                \end{itemize}
        \par Com o intuito de entender melhor nosso público alvo além de estarmos abertos à outras sugestões precisávamos de um sistema que suportasse não só uma votação fechada como também permitisse que os próprios usuários fossem capazes de adicionar outras propostas de forma dinâmica àquelas já existentes.
        \par Pensando nisso resolvemos utilizar o sistema POP (Painel de Opinião Pública) para efetuar a enquete.
        \par Foi então criada uma instância independente do sistema adaptada chamada POP-TCC (figura \ref{fig:pop-tcc}) utilizando o Heroku que poderia ser acessada pelo endereço pop-tcc.herokuapp.com. 
        \par Em 11/01 foi enviado o primeiro e-mail com a enquete do POP-TCC aberta para a lista de e-mails dos alunos do IME com as duas opções iniciais de projeto supracitadas. \par A divulgação da enquete concentrou-se principalmente via facebook nas seguintes comunidades:
\begin{figure}[htb]
\includegraphics[width=15cm]{figuras/pop-tcc}
\caption{\label{fig:pop-tcc} Sistema POP-TCC com a enquete para votação.}
\end{figure}
\begin{table}[]
\centering
\caption{Comunidades do Facebook na qual foram feitas divulgações}
\label{my-label}
\begin{tabular}{ll}
{\color[HTML]{3531FF} \textbf{Comunidade}} & {\color[HTML]{3531FF} \textbf{Número de Membros}} \\
USP - Universidade de São Paulo            &                                                   \\
FAU USP                                    & 4000                                              \\
IME USP                                    & 3000                                              \\
Universidade de São Paulo                  & 5000                                              \\
Baladas USP                                & 15000                                            
\end{tabular}
\end{table}
\par Ao longo de duas semanas outras opções de projeto surgiram. O resultado final \ref{fig:poll_chart} da enquete e a descrição das sugestões seguem abaixo:
\begin{itemize}
\item{ 1º Lugar (39 votos):} Usp avisa eventos e incidentes: Um sistema para reportar desde eventos acontecendo no campus (palestras, festas, etc) até outros incidentes (buracos, perigos, etc).
\item {2º Lugar (21 votos):} Usp doações e trocas: Um sistema voltado para os membros da comunidade que desejam fazer doações ( sejam móveis, roupas usadas, etc) para outros e aqueles dispostos a receber. organizando filas de interesse,disponibilidade e urgência.
\item {3º Lugar (20 votos):} Usp caronas: os motorizados colocam horário, bairro, quantidade de lugares disponíveis, ponte de embarque e desembarque. os interessados enviam um alerta para os motorizados confirmarem até preencherem as vagas.
\item {4º Lugar (18 votos):} Mapa de crimes na usp: um app onde roubos, furtos, assaltos, agressões, assédios, discriminações e outros crimes podem ser relatados, georreferenciados no campus.
\item {5º Lugar  (14 votos):} Volta pedalusp - volta e desenvolvimento da ideia que já teve adesão, mas morreu por falta de manutenção. implementação de novos pontos de troca de bikes mais próximos das faculdades e outros pontos estratégicos.
\item {6º Lugar (13 votos): } Usp extensão: uma plataforma de apoio mútuo para organização, contato, criação e divulgação de projetos de extensão dentro da universidade.
\item {7º Lugar (12 votos):} Usp gigabyte: um ponto de encontro virtual pra reunir o pessoal e tomar uma rodada de suco com a galera. curtição, azaração e muita gente bonita neste, que é o ponto mais quente da internet!
\item {8ºLugar  (8 votos):} Monitoria voluntária: pessoas divulgam horário e local no qual pessoas podem procurá-las para tirar dúvidas sobre certas disciplinas comuns a vários cursos.
\end{itemize}
\begin{figure}[htb]
\includegraphics[width=10cm]{figuras/poll_chart}
\caption{\label{fig:poll_chart} Gráfico gerado pelo resultado da Enquete.}
\end{figure}
\par O projeto escolhido então foi o Usp Avisa Eventos e Acidentes que seria renomeado apenas para USP Eventos.
\section{Definindo as características do Sistema}
\subsection{Pesquisa de Sistemas Semelhantes}
\par Definido o tema do projeto teve início uma pesquisa para levantar a existência de sistemas que tivessem uma proposta semelhante de divulgação de eventos e atividades.
\par Ao realizar essa pesquisa expandimos o escopo para que os sistemas não necessariamente fossem voltados para fins acadêmicos apenas tendo como objetivo principal obter uma base de conhecimento de quais funcionalidades um sistema de divulgação possui ou mesmo encontrar uma  solução em código aberto que pudesse ser utilizada como base.
\par Tal abordagem de pesquisas de concorrentes e projetos semelhantes está presente na metodologia Lean Startup cuja principal justificativa é obter o máximo de conhecimento prévio por meio de idéias já consolidadas e aplicadas por outros. (inserir referencia)
 \par Também Foi enviado um e-mail em março para a lista de Representantes Discentes do próprio IME questionando sobre funcionalidades e sugestões. 
\par  Das compilações sobre a pesquisa e resposta obtidas por e-mail destacam-se 3 sites com proposta e conteúdos semelhantes:
\begin{itemize}
\item{Eventos USP (http://www.eventos.usp.br/):} 
Canal de divulgação oficial da USP para eventos ocorrendo em suas depêndencias para todos os campus.\\
Ponto Fortes: Sendo o canal de comunicação oficial da universidade ele mostra-se com uma ótima fonte de conteúdo sobre eventos oficiais.\\
Ponto Fracos:  É uma via de mão única na qual o usuário apenas se informa dos detalhes do evento mas não tem oportunidade de criar ou divulgar o seu próprio.
\item {Catraca-Livre (https://catracalivre.com.br/brasil/):} Principal Portal de atividades culturais e divulgação de eventos costuma ser bastante eclético. \\
Pontos Fortes: Apesar de não ter um conteúdo personalizável a página de Agenda possui uma grande variedade de opções de filtro.
Pontos Fracos: Como divulga eventos por toda a cidade sua navegação é bastante confusa levando o usuário a facilmente perder-se devido à grande quantidade de informações e poucas opções de personalização. Soma-se a isso o fato de que o site é uma forma de comunicação unilateral não permitindo que usuários divulguem e organizem seus próprios eventos.\\
\item {SP Cultura (http://spcultura.prefeitura.sp.gov.br/):} Portal oficial da prefeitura para divulgação de atividades por toda a cidade.
Pontos Fortes: Possui bastante opções de filtros além de permitir que um usuário intereressado crie um evento e submeta-o para aprovação, bastante intuitivo e de fácil acessso. É baseado em uma solução de código aberto.
Pontos Fracos: Os eventos estão distribuídos por um mapa georreferenciado, dando mais enfase à localização do evento do que sobre sua descrição obrigando o usuário a selecionar primeiro um evento em uma localização para então obter informações do mesmo.
\end{itemize}
\subsection{Plataforma Web x Móvel}
\par Para decidir em qual plataforma desenvolver nosso sistema levamos em considerações fatores técnicos e do público-alvo.
\par De acordo com o gráfico \ref{fig:global_market_sharing} observa-se que as plataformas Android e IOS hoje em dia correspondem respectivamente à 86,2\%  e 12,9\% do mercado mundial de sistemas operacionais mobile totalizando juntas 99,1\% do mercado o que significaria que ao desenvolver dois aplicativos nativos para ambas as plataformas corresponderia à ter a acesso à quase totalidade do mercado móvel. \footnote{Fonte: Statista   \url{http://www.statista.com/statistics/254653/mobile-internet-user-penetration-in-brazil/}}
\begin{figure}[htb]
\includegraphics[height=10cm]{figuras/global_market_sharing}
\caption{\label{fig:global_market_sharing} Distribuição de mercado para Sistemas Mobile.}
\end{figure}
\par Foram feitas  as seguintes considerações quanto ao desenvolvimento de uma aplicação nativa. \footnote{Fonte: Caelum \url{http://blog.caelum.com.br/aplicacoes-mobile-web-ou-nativa/}}
\begin{itemize}
\item O desenvolvimento para Android é feito utilizando a linguagem Java a partir de APIs fornecidas pelo próprio Google enquanto um aplicativo para IOS utiliza Objective-c ou Swift por meio das APIs fornecidadas pela Apple. Devido a isso não há correspondencia de código entre ambas plataformastornando necessário o desenvolvimento de duas aplicações distintas caso queira-se atingir a totalidade do mercado.
\item Os aplicativos nativos seguem um padrão bastante rígido de UI/UX que divergem bastante entre si. Os padrões de desenvolvimento para interfaces  de um aplicativo Android diverge totalmente de um aplicativo IOS  quanto a sua usabilidade e layout fazendo com que seja necessário pensar em duas interfaces distintas.
\item A maior vantagem de um aplicativo nativo é ter acesso aos recursos de \emph{hardware} do smartphone tais como câmera ou acelerômetro. Como nossa aplicação não faria uso de nenhum recurso específico tais vantagens não seriam aproveitadas.
\item Os aplicativos nativos estão sujeitos às normas e aprovações de suas lojas virtuais, Play Store para  o Android e Apple Store para o IOS, fazendo com que o tempo de publicação de uma atualização  aumente devido a necessidade de aprovação da loja em questão.
\item Não seria possível utilizar a plataforma em Desktops restringindo o público alvo.
\end{itemize}
\par Para tomar a decisão de desenvolver uma plataforma web foi levado em consideração principalmente a facilidade de atualização de uma aplicação web por não necessitar da aprovação de um loja online devido a necessidade de desenvolvimento segundo o modelo proposto pelo modelo \emph{Lean Startup} e possibilidade de acesso por meio de diversas plataformas (tanto via desktop como móveis) com uma única versão da aplicação pois ainda que exista o custo de abrir mão de uma interface específica com a identidade visual de cada plataforma demos prioridade a esse modelo pro atingir um número maior de usuários.
\par Apesar da escolha de um sistema web ao analisar o crescimento do acesso móvel no Brasil~\ref{fig:mobile_internet} que vem aumentando a passos largos no país houve a preocupação de desenvolver-se uma aplicação web híbrida com uma interface totalmente responsiva \footnote{ Uma interface responsiva de um site ou página é uma versão do layout adaptada para uso em telas menores comumente refere-se a visualização em smartphones} desde o princípio.
\begin{figure}[htb]
\includegraphics[height=10cm]{figuras/mobile_internet}
\caption{\label{fig:mobile_internet} Crescimento do Uso de internet móvel no Brasil.}
\end{figure}
\par Dessa forma apesar do acesso em um smartphone não ser tão intuitivo quanto em uma aplicação nativa ainda sim teria uma interface funcional garantindo que a navegação em uma plataforma móvel fosse feita sem  dificuldades.
\subsection{Porque escolher Ruby on Rails }
\par O uso do Ruby on Rails entre startups tem crescido  nos últimos anos isso devido a alguns fatores intrisicamente ligados a própria estrutura do arcabouço rails e também à necessidade que startups e o próprio modelo de \emph{Lean Startup} exigem para iterações e alterações de códigos rápidas.
\par Uma das principais filosofias do Rails, o CoC \- Convenção sobre Configuração, permite que o desenvolvedor possa, a partir de um conjunto pré-definido de configurações padrões, agilizar o desenvolimento do código ao tirar de sua responsabilidade fatores de configuração em detrimento de um padrão já estabelecido  dando assim ao desenvolvedor mais tempo para concentrar-se em decisões sobre o produto. Essa agilidade em codificar rapidamente resulta em iterações mais rápidas contribuindo para uma maior agilidade dentro do ciclo de Contruir-Medir-Aprender.
\par Outro fator inerente a cultura Rails é o grande enfoque em testes automatizados. Para o desenvolvimento da plataforma USP evento foram utilizadas as  gems Rspec e Capybara que permitem não só  a realização de testes unitários como também testes de integração de modo muito rápido e direto garantindo assim a confiabilidade  do seu projeto evitando que um \emph{bug} imprevisto prejudique seu aprendizado sobre o MVP. 
\par Soma-se a isso a quantidade enorme de ferramentas que auxiliam na integração contínua do código como por exemplo o Travis CI utilizado  durante o desevolvimento da plataforma. 
\par O Travis CI era responsável para que a cada commit realizado para o repositório principal do projeto fosse executada toda a bateria de testes automatizados enviando um e-mail contendo um relatório sobre o seu resultado, inclusive em caso de falha, garantindo dessa forma que todos tivessem sempre conhecimento das alterações sobre o código além de evitar  problemas de conflitos de versões ou mesmo que uma atualização de projeto fosse colocada no ambiente de produção contendo alguma falha.
\par O fator comunidade é outra vantagem do  RoR. Destacando-se pelo seu tamanho, interesse e acessibilidade em tirar duvidas sempre contribuindo para promover o arcabouço  a comunidade que orbita ao redor do Rails foi capaz de criar ferramentas prontas para uso com uma ótima documentação, tutoriais, cursos e guias garantindo assim que qualquer desenvolvedor interessado sempre tenha em mãos materiais e ferramentas de qualidade para iniciar o desenvolvimento do seu projeto.
\par Acrescenta-se também à favor do arcabouço Rails sua escalabilidade, rails é utilizado por empresas de grande porte  tais como Groupon, Twitter, Basecamp, mostrando que o arcabouço é um arcabçou robusto capaz de lidar com grandes sistemas  sem ter uma queda de desempenho.\footnote{Fonte: MLSDev Why startups use Ruby on Rails? \url{http://mlsdev.com/en/blog/61-why-startups-use-ruby-on-rails}}
\par Para finalizar Rails é também um arcabouço seguro garantindo mesmo em sua configuração padrão ou com o auxilio de alguns plugins proteção contra SQL-Injections e XSS( \emph{Cross Site Scripting}). Além disso os programadores devem seguir o \emph{Secure Life Cycle Development}, figura  \ref{fig:security_dev_lifecycle}, proposto pela Microsoft, um modelo de desenvolvimento de software cujo principal objetivo é ajudar o a construir software mais seguros e confiáveis e reduzir custos \footnote{Fonte: Wikipedia \url{https://en.wikipedia.org/wiki/Microsoft_Security_Development_Lifecycle}}.
\begin{figure}[htb]
\includegraphics[width=15cm]{figuras/security_dev_lifecycle}
\caption{\label{fig:security_dev_lifecycle} Fases do Secure Life Cycle Development.}
\end{figure}
\section{Kanban}
\par Kanban é uma palavra japonesa e significa cartão visual para prover informação de regulamentação do fluxo de estoque e materiais. Possui três regras principais (KNIBERG, H., 2009): visualizar o fluxo de trabalho; limitar o trabalho em cada estágio do fluxo, e medir o tempo de avanço (tempo médio para se completar cada item) (inserir referência).
\par No contexto de Métodos Ágeis e \emph{Lean Startup} foi utilizado uma abordagem utilizando o Kanban para definir as hipóteses de produtos,  o escopo de desenvolvimento e as tarefas para serem executadas durante as etapas de desenvolvimento.
\par Tradicionalmente o kanban para desenvolvimento de Software possui 3 estágios:
\begin{itemize}
        \item TO DO: referente a requisitos que ainda estão aguardando para serem
desenvolvidos.
        \item DOING: referente a requisitos que estão sendo desenvolvidos.
        \item DONE. referente a requisitos que já finalizaram e foram devidamente
revisados e testados. (inserir referência).
\end{itemize}
\par Cada item adicionado na fila de TO DO é chamada de tarefa. Uma hipótese à ser testada pode ser transformada em uma série de tarefas pequenas à serem completadas durante o período de desenvolvimento.
\par Caso seja necessário é possível quebrar uma tarefa grande em uma série de tarefas menores. Tomando como exemplo a tarefa de Implementar um filtro de eventos para a página principal. Ela foi quebrada em 3 tarefas menores para serem completadas: Implementar Categorias de Eventos, Permitir adicionar categorias de eventos durante a criação de um novo evento e Implementar filtro de Eventos na página principal de eventos.
\par O tempo que uma tarefa demora desde sua entrada no quadro até a saída é denominado \emph{Lead Time}. Em um ambiente de Startup o intuito é sempre obter o menor Lead Time possível para tal é importante estar ciente da taxa de entrega que sua equipe de desenvolvimento é capaz de cumprir e sempre definir as tarefas de modo simples. 
\par Para utililzação no projeto USP Eventos foi incluída uma coluna a mais denominada BACKLOG, comumente usada na metodologia Scrum \footnote{Fonte: Desenvolvimento Ágil \url{fonte http://www.desenvolvimentoagil.com.br/scrum/sprint_backlog}},  na qual foram colocadas idéias que poderiam ou não serem transformadas em tarefas de desenvolvimento ou hipóteses para serem executadas em uma iteração do ciclo de Construir-Medir-Aprender.
\par Em cada iteração do ciclo especificamos quais hipóteses seriam testadas e a partir delas criamos tarefas para serem implementadas.
\par Caso um bug fosse detectado uma tarefa era cridada na fila de TO DO para resolvê-la.
\par A ferramenta Trello, figura \ref{fig:kanban}, foi utilizada para  simular um quadro Kanban digital, com ela foi possível guiar todo o desenvolvimento do sistema.
\begin{figure}[htb]
\includegraphics[width=15cm]{figuras/kanban}
\caption{\label{fig:kanban} Sistema Trello contendo as 4 colunas utilizadas no USP Eventos.}
\end{figure}
        Com o intuito de tornar melhor a visualização no Trello foram criados alguns Rótulos de cores distintas para cada tarefa:
        \begin{itemize}
        \item BUG (vermelho): Defeito não previsto durante o desenvolvimento.
        \item FEATURE (verde): Nova funcionalidade para ser imeplementada.
        \item REFACTOR(lilás):  Melhoria de código sem refletir uma mudança externa.
        \item MELHORIAS(laranja): Investigar ou implementar o uso de ferramentas externas ao sistema, tal qual a integração com o Google Analytics por exemplo.
        \item QUICK WIN(amarelo): Melhoria feita rapidamente e não prevista durante a definição de tarefas.
        \end{itemize}   
\section{Primeira Iteração}
\subsection{Construção}
< em desenvolvimento>
         \par O primeiro MVP do USP Eventos tinha como objetivo testar as seguintes hipóteses:
         \begin{itemize}
         \item Medir o interesse do público em participar de um evento.
         \item Criar uma interface intuitiva e rápida para mostrar informações de eventos para o usuário.
         \end{itemize}
        \par A Página inicial (figura \ref{fig:landing_pagev1}) do site possuia acesso para a página de cadastro e login além de um formulário para envio de sugestões.
        \begin{figure}[htb]
		\includegraphics[width=15cm]{figuras/landing_pagev1}
		\caption{\label{fig:landing_pagev1} Tela inicial na primeira iteração }
		\end{figure}
        \par O cadastro de usuário (figura \ref{fig:sign_upv1}) pedia apenas nome, e-mail e senha inicialmente porém ainda na primeira iteração foi implementada a opção de login com facebook.
        \begin{figure}[htb]
		\includegraphics[width=15cm]{figuras/sign_upv1}
		\caption{\label{fig:sign_upv1} Tela de Cadastro na primeira iteração}
		\end{figure}
		\par A página de eventos (figura \ref{fig:events_pagev1}) só poderia ser acessada por um usuário logado tornando essa página e qualquer página de evento específica inacessível para um visitante sem login.
        \par Além disso a página de eventos apenas mostrava-os  sem oferecer qualquer opção inicial de filtro.
        \begin{figure}[htb]
		\includegraphics[width=15cm]{figuras/events_pagev1}
		\caption{\label{fig:events_pagev1} Página de Eventos}
		\end{figure}
\par Todas as páginas foram pensadas também para o acesso via mobile possuindo versões responsivas ( figura \ref{fig:events_pagev1_responsive}).
        \begin{figure}[htb]
		\includegraphics[width=5cm]{figuras/events_pagev1_responsive}
		\caption{\label{fig:events_pagev1_responsive} Página de Eventos versão responsiva}
		\end{figure}        
\par Visando evitar que os usuários se deparassem com uma página de eventos vazia foi feita uma \emph{Rake Task} \footnote{Rake é um programa implementado em \emph{Ruby} que permite ao usuário implementar \emph{tasks} que são executadas ao serem chamadas} para consumir o xml gerado pelo feed RSS que o site www.eventos.usp.br. Dessa forma seria possível adicionar de forma mais ágil alguns eventos dentro da plataforma.
\par Cada thumbnail de eventos presente na página principal de listagem de eventos incluía o nome do evento, localização, data de início e fim além de um botão de "Participar" para que os usuários que estivessem logados e também botões para compartilhamento nas principais redes sociais.
\par Houve também a preocupação de espalhar formulários de Sugestões em diversos pontos do site com o intuito de facilitar a coleta de informações do usuário.
\subsection{Divulgação}
\par A primeira versão do sistema ficou disponível apartir do dia 5 de maio e sua divulgação foi feita pelo facebook por grupos e comunidades associadas a institutos da USP tais como FFLCH, FAU, IME, ECA, assim como foram enviadas mensagens para as respectivas empresas Júnior e Atléticas.
\par Também divulgamos a primeira versão na lista de alunos e representantes discentes do próprio Instituto de Matemática e Estatística.
\subsection{Métricas}
\par Para medir o fluxo de usuários dentro do site utilizamos o Google Analytics em conjunto com o Google Tag Manager.
\par Como métrica chave escolhemos medir a quantidade de usuários que se interessavam por um evento.
\par Foi criada então uma tag (figura \ref{fig:tags}) para rastrear os cliques no botão "Participar" presente dentro do thumbnail de um evento na página principal  e também na página de detalhes do evento. Dessa forma seria possível mapear o interesse do usuário em um determinado evento.
\begin{figure}[htb]
\includegraphics[width=15cm]{figuras/tags}
\caption{\label{fig:tags} Visualização das Tags pelo Google Tag Manager.}
\end{figure}
\par Em pararelo foi possível obter também através do google analytics as seguintes  informações (figura \ref{fig:analytics_1interacao_dados}):
\begin{itemize}
\item Visualizações de Páginas: número total de páginas visualizadas no período.
\item Páginas / Sessão: Número de páginas acessadas por sessão em média
\item Duração Média da Sessão: auto-explicativo.
\item Usuários: Toma-se como usuário o número de acessos únicos realizados.
\item Porcentagem de Saída
\end{itemize}

\begin{figure}[htb]
\includegraphics[width=15cm]{figuras/analytics_1interacao_dados}
\caption{\label{fig:analytics_1interacao_dados} Dados obtidos pelo G.A de 01/05 até 31/07}
\end{figure}

\par É possível observar uma grande taxa de rejeição inicial ao site no período associado principalmente na página inicial com cerca de 761 sessões e 423 desistências (figura \ref{fig:analytics_1ainteracao_fluxo}).
\par Durante o períogo observado foram registrados apenas 184 cliques em 56 sessões únicas no botão "Participar".
\begin{figure}[htb]
\includegraphics[width=15cm]{figuras/analytics_1ainteracao_fluxo}
\caption{\label{fig:analytics_1ainteracao_fluxo} Fluxo de Comportamento de 01/05 até 31/07.}
\end{figure}

\par Através do gráfico de a divisão de uso por tipo de Sistema Operacional (figura \ref{fig:analytics_1interacao_so})foi possível observar que a grande maioria dos usuários acessa o site por meio \emph{notebooks} ou computadores pessoais.
\begin{figure}[htb]
\includegraphics[width=15cm]{figuras/analytics_1interacao_so}
\caption{\label{fig:analytics_1interacao_so} Porcentagem de uso por S.O de 01/05 até 31/07.}
\end{figure}
\subsection{Aprendizado}
\par  O maior número de acesso de usuários deu-se sempre em seguida aos posts realizados pelo facebook, alcançando picos de acesso mostrando a importância da divulgação pela plataforma.
\par Foi observado que os usuários acessavam o site porém não realizavam cadastro deixando-o logo em seguida resultando em um número elevado de desistências na página inicial. 
\par Dentre os retornos recebidos através do formulário do site, e-mails e de forma direta foram compiladas algumas críticas e sugestões:
\begin{itemize}
\item Página de eventos e visualização dos mesmos deveriam ser abertas para usuários mesmo sem login.
\item Ausência de um filtro de usuários tornou a página de eventos confusa para navegação.
\item Botão lateral de adicionar evento estava muito grande e atrapalhando a navegação.
\item Falta de cores na pagina principal tornou cansativa a navegação.
\item Clicar no nome do do evento no thumbnail para acessar a página do mesmo.
\item Ausência de opção de "esqueci minha senha".
\item Clicar no botão "Participar" não tinha uma utilidade prática. O evento era salvo porém isso gerava nenhum reflexo no sistema não existindo uma funcionalidade que justificasse sua existência e não havendo razão para que os usuários clicassem no botão.
\par Alguns comentários selecionados:

\par Por Veronica: "Seria muito legal poder filtrar os eventos por tags referentes ao local / tipo / assuntos que serão abordados!"

\par Por Lucas: "Olá! Eu gostaria de ver os eventos por categoria/área de conhecimento (Artes, História, Economia, Engenharia, etc)."

\par Por Carolina: "Por que devemos nos cadastrar simplesmente para acessar o site? E se a pessoa "simplesmente quer se informar sobre o que está acontecendo"? A minha sugestão é que somente quem quer enviar eventos para o site deveria se ter que se cadastrar. Obrigada e boa sorte no TCC."

\par Por Karina: ". [Login] Aos usuários que não possuem conta, mas tentam logar, seria ideal que o sistema mostra-se quando o usuário colocou dados incorretos e quando o usuário não possui conta. 
\\
. [Página de eventos] Seria melhor que a célula do evento permitisse o click para adentrar detalhes sobre o mesmo
\\
. [Página de eventos] Inserir algumas ferramentas de filtro: datas (inicio/final ou datas pontuais); tags (algumas tags pré-cadastradas); campus;
\\
. [Página de eventos] Espaço para uma imagem nas celulas de divulgação do evento daria mais cor e chamaria mais a atenção dos usuários
\\
. Página de eventos] Seria legal colocar um aviso de inscrições limitadas para eventos que têm tal restrição"

\end{itemize}
\section{Segunda Iteração}
\subsection{Construção}
\par Levando em consideração o aprendizado da primeira iteração foi feita uma mudança no fluxo do site para permitir o acesso para a página de eventos sem a necessidade de realizar um cadastro antes ou exigir um \emph{login} do usuário.
\par As hipóteses a serem testadas foram:
\begin{itemize}
\item Verificar se as alterações visuais foram bem aceitas
\item Testar a hipótese da necessidade de Filtro para Eventos
\end{itemize}
\par Foi criado um filtro para a página de eventos baseado na utilização de tags dessa forma ao criar um novo evento (figura \ref{fig:event_newv2} )o usuário agora pode escolher 3 dentre 12 tags pré-definidas que servirão como filtro nas página principal.
        \begin{figure}[htb]
		\includegraphics[width=15cm]{figuras/event_newv2}
		\caption{\label{fig:event_newv2} Página de Cadastro de Novos eventos com filtros}
		\end{figure} 

\par Visando tornar a navegação dentro da página de eventos mais fluída e menos cansativa fora realizadas algumas modificações visuais na exibição dos eventos (figura  \ref{fig:events_pagev2})
\begin{itemize}
\item Remodelagem do thumbnail de Eventos
\item Diminuição do botão de adicionar novos eventos para não atrapalhar a navegação
\item Adição de um menu lateral com opções de Filtros para os eventos
\item Nova listagem personalizada de Eventos segundo os interesses do usuário
\end{itemize}
        \begin{figure}[htb]
		\includegraphics[width=15cm]{figuras/events_pagev2}
		\caption{\label{fig:events_pagev2} Página de Eventos com as alterações para a segunda interação}
		\end{figure}

\par Os filtros (figura \ref{fig:events_pagev2_filters}) estão em um menu lateral que é acionado por um botão na parte superior esquerda. Também é possível selecionar cada \emph{tag} individualmente ao clicar sobre a respectiva nos \emph{thumbnails} de eventos.
        \begin{figure}[htb]
		\includegraphics[width=15cm]{figuras/events_pagev2_filters}
		\caption{\label{fig:events_pagev2_filters} Filtros na página de Eventos}
		\end{figure}
As mudanças realizadas (figura \ref{fig:thumbs_comparison_v1_v2}) no thumbnail de eventos:
\begin{itemize}
\item Removido botão de +Info, agora para acessar mais informações basta clicar sobre o nome do evento
\item Adicionado Cabeçalho para separar e dar maior ênfase para o título e uma cor de fundo para aumentar o contraste com o plano de fundo a fim de facilitar a leitura
\item Adição de tags com as classificações dos eventos facilitando a escolha do mesmo.
\item Mudança do nome do botão de "Participar" para "Salvar"
\end{itemize}
        \begin{figure}[htb]
		\includegraphics[width=15cm]{figuras/thumbs_comparison_v1_v2}
		\caption{\label{fig:thumbs_comparison_v1_v2} Esquerda: versão antiga. Direita: Versão atualizada}
		\end{figure} 

\par Em conjunto com a criação das tags para eventos foi criado um mecanismo de preferências para o usuário, agora na página de cadastro ou edição de usuário é possível selecionar as tags com as quais o usuário tenha maior afinidade com o intuito de exibir uma listagem personalizada de eventos segundo esses critérios.
\par Aproveitando o sucesso do lançamento do aplicativo Pokemon GO \footnote{Fonte: Pokemon GO \url{ https://en.wikipedia.org/wiki/Pokemon_Go }} que gerou uma enorme quantidade de acessos e movimentação pelo Campus foi criado uma página  de Alertas \ref{fig:alert_page} no Campus visando atingir o público que estava utilizando o aplicativo para divulgar a localização de Pokemons.
        \begin{figure}[htb]
		\includegraphics[width=15cm]{figuras/alert_page}
		\caption{\label{fig:alert_page} Página de Alertas}
		\end{figure} 
\par A implementação de um testes para a página de Alertas tão rapidamente só foi possível devido a grande flexibilidade que o modelo de Construir-Medir-Aprender oferece pois somente assim foi possível rapidamente integrar uma nova funcionalidade não prevista  dentro do escopo do projeto e medir sua eficácia.
\par Além disso com o auxílio de testes automatizados, ferramentas para integração contínua e a agilidade do desenvolvimento em Rails foi possível desenvolver e colocar as alterações no ambiente de produção sem comprometer a integridade do sistema como um todo.

\par Por fim foi adicionada uma página "Sobre" com informações sobre os responsáveis pelo site assim como seus objetivos.
\subsection{Divulgação}
\par Além da divulgação pelo facebook foram espalhados cartazes em pontos estratégicos pela USP tais como Pontos de Ônibus com grande movimentação, murais próximos aos Bandejões e também no interior de alguns institutos.

<fotos cartazes no pontos de onibus>

\subsection{Métricas}

\par Foi mantido o rastreamento pelo google Analytics do botão "Participar' porém seu nome foi alterado para "Salvar" com o intuito refletir melhor sua utilidade: Salvar um evento como interessante para exibi-lo na seção de "Meus Eventos" da listagem de eventos do usuário.

\par A nossa métrica chave continuou sendo avaliar o interesse dos usuários em determinado Evento por meio do clique no botão "Salvar".

\par Pelos resultados obtidos através do Googgle Analytics ((figura \ref{fig:analytics_2ainteracao_dados}) foi observado uma diminuição no número de sessões entretando também houve uma diminuição significativa na taxa de rejeição do site, caindo de 46,91\% para 33,16\%. 
\par o tempo médio por sessão também aumentou pasasndo de 1:54 minutos para 3:37 minutos. Mostrando um aumento na retenção de usuários acessando a plataforma.
\begin{figure}[htb]
\includegraphics[width=15cm]{figuras/analytics_2ainteracao_dados}
\caption{\label{fig:analytics_2ainteracao_dados} Dados obtidos pelo G.A de 01/08 até 30/09}
\end{figure}

\par Com o lançamento da seção de Alertas foi feita uma divulgação via Facebook incentivando os usuários a utilizarem a plafaforma com o intuito de divulgar a localização de Pokemons.
\par Foi colocado também uma tag para rastrear o número de clicks no botão "Alertas" com a intenção de medir o interesse na funcionalidade, dessa forma foram observados 305 cliques no botão "Alertas" no período.

\par Analisando o o gráfico de divisão por tipo de Sistema Operacional ((figura \ref{fig:analytics_2ainteracao_so}) percebemos que o acesso pelo sistema Android proporcionalmente mais que dobrou em relação ao período anterior, passando de 8,4\% para 19,36\%.

\begin{figure}[htb]
\includegraphics[width=15cm]{figuras/analytics_2ainteracao_so}
\caption{\label{fig:analytics_1interacao_so} Porcentagem de uso por S.O de 01/08 até 30/09.}
\end{figure}

\subsection{Aprendizado}

\par Com a abertura da página de eventos sem a obrigatoriedade de um cadastro mais usuários acessaram a página principal de Eventos porém quase não houveram novos cadastros dificultando assim que um usuário salvasse algum evento para sua lista ou tivesse algum incentivo para retornar ao site.

\par Com a divulgação por cartazes foi possível constatar um aumento na utilização em dispositivos móveis principal forma de acesso em lugares públicos e incentivada devido ao QR Code presente nos cartazes.

\par Apesar do pico de acessos com o lançamento da página de Alertas passado o período de pico a funcionalidade foi abandonada pelos usuários, gerando poucos acessos mostrando que talvez não fosse interessante investir em seu desenvolvimento.

\par Mesmo com a criação e filtros e melhorias visuais a página principal ainda carecia de apelo para navegação. 
\par Alguns comentários recebidos de forma oral afirmaram que a página de eventos estava pouco atrativa visualmente sendo necessário que ela tivesse mais elementos que prendessem a atenção do usuário. 

\par Novamente recebemos comentários pelo próprio formulário do site sobre adicionar a opção de incluir uma foto ao evento,  segue: "Sinto falta de anexo para cartazes. poder enviar uma foto ou cartaz escaneado do evento." - Ferdinand Machado
